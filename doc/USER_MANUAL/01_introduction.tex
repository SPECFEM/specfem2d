%------------------------------------------------------------------------------------------------%

\chapter{Introduction}

%------------------------------------------------------------------------------------------------%

SPECFEM2D allows users to perform 2D and 2.5D (i.e., axisymmetric) simulations of
acoustic, elastic, viscoelastic, and poroelastic seismic wave propagation
as well as full waveform imaging (FWI) or adjoint tomography.\newline

In fluids,
%% DK DK removed that because gravity is implemented in SPECFEM3D but not in SPECFEM2D
% when gravity is turned off,
SPECFEM2D uses the classical linearized Euler equation; thus if you have sharp local variations of density in the fluid (highly heterogeneous fluids
in terms of density) or if density becomes extremely small in some regions of your model (e.g. for upper-atmosphere studies), before using the code please make sure the linearized Euler equation is a valid approximation in the case you want to study. For more details on that see e.g. \cite{COA2011}.\newline

The 2D spectral-element solver accommodates
regular and unstructured meshes, generated for example by Cubit
\urlwithparentheses{http://cubit.sandia.gov},
Gmsh \urlwithparentheses{http://geuz.org/gmsh}
or GiD \urlwithparentheses{http://www.gid.cimne.upc.es}.
Even mesh creation packages that generate triangles, for instance Delaunay-Voronoi triangulation codes, can be used because each triangle can then easily be decomposed into three quadrangles by linking the barycenter to the center of each edge; while this approach does not generate quadrangles of optimal quality, it can ease mesh creation in some situations and it has been shown that the spectral-element method can very accurately handle distorted mesh elements.\newline

With version 7.0, the 2D spectral-element solver accommodates Convolution PML absorbing layers and well as higher-order time schemes
(4th order Runge-Kutta and LDDRK4-6).
Convolution or Auxiliary Differential Equation Perfectly Matched absorbing Layers (C-PML or ADE-PML)
are described in \cite{MaKoEz08,MaKoGe08,MaKo09,MaKoGeBr10,KoMa07}.\newline

The solver has adjoint capabilities and can
calculate finite-frequency sensitivity kernels \citep{TrKoLi08,PeKoLuMaLeCaLeMaLiBlNiBaTr11} for acoustic,
(an)elastic, and poroelastic media. The package also considers 2D SH
and P-SV wave propagation. Finally, the solver can run
both in serial and in parallel. See SPECFEM2D
\urlwithparentheses{https://github.com/SPECFEM/specfem2d}
for the source code.\newline

The SEM is a continuous Galerkin technique \citep{TrKoLi08,PeKoLuMaLeCaLeMaLiBlNiBaTr11}, which can easily be made discontinuous \citep{BeMaPa94,Ch00,KoWoHu02,ChCaVi03,LaWaBe05,Kop06,WiStBuGh10,AcKo11}; it is then close to a particular case of the discontinuous Galerkin technique \citep{ReHi73,LeRa74,Arn82,JoPi86,BoMaHe91,FaRi99,HuHuRa99,CoKaSh00,GiHeWa02,RiWh03,MoRi05,GrScSc06,AiMoMu06,BeLaPi06,DuKa06,DeSeWh08,PuAmKa09,WiStBuGh10,DeSe10,EtChViGl10}, with optimized efficiency because of its tensorized basis functions \citep{WiStBuGh10,AcKo11}.
In particular, it can accurately handle very distorted mesh elements \citep{OlSe11}.\newline

It has very good accuracy and convergence properties \citep{MaPa89,SePr94,DeFiMu02,Coh02,DeSe07,SeOl08,AiWa09,AiWa10,MeStTh12}.
The spectral element approach admits spectral rates of convergence and allows exploiting $hp$-convergence schemes.
It is also very well suited to parallel implementation on very large supercomputers \citep{KoTsChTr03,TsKoChTr03,KoLaMi08a,CaKoLaTiMiLeSnTr08,KoViCh10}
as well as on clusters of GPU accelerating graphics cards \citep{Kom11,MiKo10,KoMiEr09,KoErGoMi10}.
Tensor products inside each element can be optimized to reach very high efficiency \citep{DeFiMu02}, and mesh point and element numbering can be optimized to reduce processor cache misses and improve cache reuse \citep{KoLaMi08a}. The SEM can also handle triangular (in 2D) or tetrahedral (in 3D) elements \citep{WinBoyd96,TaWi00,KoMaTrTaWi01,Coh02,MeViSa06} as well as mixed meshes, although with increased cost and reduced accuracy in these elements, as in the discontinuous Galerkin method.\newline

Note that in many geological models in the context of seismic wave propagation studies
(except for instance for fault dynamic rupture studies, in which very high frequencies or supershear rupture need to be modeled near the fault, see e.g. \cite{BeGlCrViPi07,BeGlCrVi09,PuAmKa09,TaCrEtViBeSa10})
a continuous formulation is sufficient because material property contrasts are not drastic and thus
conforming mesh doubling bricks can efficiently handle mesh size variations \citep{KoTr02a,KoLiTrSuStSh04,LeChLiKoHuTr08,LeChKoHuTr09,LeKoHuTr09}.\newline

For a detailed introduction to the SEM as applied to regional
seismic wave propagation, please consult \citet{PeKoLuMaLeCaLeMaLiBlNiBaTr11,TrKoLi08,KoVi98,KoTr99,ChKoViCaVaFe07} and
in particular \citet{LeKoHuTr09,LeChKoHuTr09,LeChLiKoHuTr08,GoAmTaCaSmSaMaKo09,WiKoScTr04,KoLiTrSuStSh04}.
A detailed theoretical analysis of the dispersion
and stability properties of the SEM is available in \citet{Coh02}, \citet{DeSe07}, \citet{SeOl07}, \citet{SeOl08} and \citet{MeStTh12}.\newline

The SEM was originally developed in computational fluid dynamics \citep{Pat84,MaPa89}
and has been successfully adapted to address problems in seismic wave propagation.
Early seismic wave propagation applications of the SEM, utilizing Legendre basis functions and a
perfectly diagonal mass matrix, include \cite{CoJoTo93}, \cite{Kom97},
\cite{FaMaPaQu97}, \cite{CaGa97}, \cite{KoVi98} and \cite{KoTr99},
whereas applications involving Chebyshev basis functions and a non-diagonal mass matrix
include \cite{SePr94}, \cite{PrCaSe94} and \cite{SePrPr95}.
In the Legendre version that we use in SPECFEM the mass matrix is purposely slightly inexact but diagonal (but can be made exact if needed, see \cite{Teu15}),
while in the Chebyshev version it is exact but non diagonal.\newline

Beware that, in a spectral-element method, some spurious modes (that have some similarities with classical so-called "Hourglass modes" in finite-element techniques,
although in the SEM they are not zero-energy modes) can appear in some (but not all) cases in the spectral element in which the source is located.
Fortunately, they do not propagate away from the source element.
However, this means that if you put a receiver in the same spectral element as a source, the recorded signals may in some cases be wrong, typically exhibiting some spurious
oscillations, which are often even non causal.
If that is the case, an easy option is to slightly change the mesh in the source region in order to get rid of these Hourglass-like spurious modes,
as explained in \cite{DuLiScGa14}, in which this phenomenon is described in details, and in which practical solutions to avoid it are suggested.\newline

All SPECFEM2D software is written in Fortran2003 with full portability
in mind, and conforms strictly to the Fortran2003 standard. It uses
no obsolete or obsolescent features of Fortran. The package uses
parallel programming based upon the Message Passing Interface (MPI)
\citep{GrLuSk94,Pac97}.\newline

This new release of the code includes support for GPU graphics card acceleration \citep{Kom11,MiKo10,KoMiEr09,KoErGoMi10}.\newline

The code uses the plane strain convention when the standard P-SV equation case is used, i.e.,
the off-plane strain $\epsilon_{zz}$ is zero by definition of the plane strain convention but the off-plane stress $\sigma_{zz}$ is not equal to zero,
one has $\sigma_{zz} = \lambda (\epsilon_{xx} + \epsilon_{yy})$.
This implies, as in any plain strain software package, that the P-SV source is a line source along the direction perpendicular to the plane (see file
discussion\_of\_2D\_sources\_and\_approximations\_from\_Pilant\_1979.pdf for more details).

%------------------------------------------------------------------------------------------------%
\section{Citation}
%------------------------------------------------------------------------------------------------%

You can find all the references below in \BibTeX format in file \texttt{doc/USER\_MANUAL/bibliography.bib}.\newline

If you use this code for your own research, please cite at least one article
written by the developers of the package, for instance:
%
\begin{itemize}
\item \cite{TrKoLi08},
\item \cite{PeKoLuMaLeCaLeMaLiBlNiBaTr11},
\item \cite{VaCaSaKoVi99},
\item \cite{LeChKoHuTr09},
\item \cite{LeChLiKoHuTr08},
\item \cite{LeKoHuTr09},
\item \cite{KoErGoMi10},
\item \cite{KoMiEr09},
\item \cite{LiPoKoTr04},
\item \cite{ChKoViCaVaFe07},
\item \cite{KoVi98},
\item \cite{KoTr99},
\item \cite{KoLiTrSuStSh04},
\item \cite{MoTr08},
\item \cite{BlKoChLoXi16},
\item and/or other articles from \url{https://specfem.github.io/komatitsch.free.fr/publications.html}.
\end{itemize}
%
If you use the C-PML absorbing layer capabilities of the code, please cite at least one article
written by the developers of the package, for instance:
%
\begin{itemize}
\item \cite{XiKoMaMa14},
\item \cite{XiMaCrKoMa16}.
\end{itemize}
%
If you use the UNDO\_ATTENUATION option of the code in order to produce full anelastic/viscoelastic sensitivity kernels, please cite at least one article
written by the developers of the package, for instance (and in particular):
%
\begin{itemize}
\item \cite{KoXiBoPeSaLiTr16}.
\end{itemize}
%
More generally, if you use the attenuation (anelastic/viscoelastic) capabilities of the code, please cite at least one article
written by the developers of the package, for instance:
%
\begin{itemize}
\item \cite{KoXiBoPeSaLiTr16},
\item \cite{BlKoChLoXi16}.
\end{itemize}
%
If you use the kernel capabilities of the code, please cite at least one article
written by the developers of the package, for instance:
%
\begin{itemize}
\item \cite{TrKoLi08},
\item \cite{PeKoLuMaLeCaLeMaLiBlNiBaTr11},
\item \cite{LiTr06},
\item \cite{MoLuTr09}.
\end{itemize}
%
If you use the SCOTCH / CUBIT non-structured capabilities, please cite:
%
\begin{itemize}
\item \cite{MaKoBlLe08}.
\end{itemize}
%
If you use axisymmetric geometries please also cite:
\begin{itemize}
 \item \cite{BoCrKoAs16}
\end{itemize}
%
The corresponding Bib\TeX{} entries may be found
in file \texttt{doc/USER\_MANUAL/bibliography.bib}.



%------------------------------------------------------------------------------------------------%
\section{Support}
%------------------------------------------------------------------------------------------------%

This material is based upon work supported by the USA National Science
Foundation under Grants No. EAR-0406751 and EAR-0711177, by the French
CNRS, French Inria Sud-Ouest MAGIQUE-3D, French ANR NUMASIS under
Grant No. ANR-05-CIGC-002, and European FP6 Marie Curie International
Reintegration Grant No. MIRG-CT-2005-017461.
Any opinions, findings, and conclusions or recommendations expressed in this material are
those of the authors and do not necessarily reflect the views of the
USA National Science Foundation, CNRS, Inria, ANR or the European
Marie Curie program.


