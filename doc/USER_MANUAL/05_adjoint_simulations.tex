%------------------------------------------------------------------------------------------------%

\chapter{Adjoint Simulations}

%------------------------------------------------------------------------------------------------%


%------------------------------------------------------------------------------------------------%
\section{How to obtain finite sensitivity kernels}
%------------------------------------------------------------------------------------------------%

\begin{enumerate}
\item Run a forward simulation:
\begin{itemize}
\item \texttt{SIMULATION\_TYPE = 1}
\item \texttt{SAVE\_FORWARD = .true.}
\item \texttt{seismotype = 1} (we need to save the displacement fields to later on derive the
adjoint source. Note: if the user forgets it, the program corrects it when reading the proper
\texttt{SIMULATION\_TYPE} and \texttt{SAVE\_FORWARD} combination and a warning
message appears in the output file)
\end{itemize}

Important output files (for example, for the elastic case, P-SV waves):
\begin{itemize}
\item \texttt{absorb\_elastic\_bottom*****.bin}
\item \texttt{absorb\_elastic\_left*****.bin}
\item \texttt{absorb\_elastic\_right*****.bin}
\item \texttt{absorb\_elastic\_top*****.bin}
\item \texttt{lastframe\_elastic*****.bin}
\item \texttt{AA.S****.BXX.semd}
\item \texttt{AA.S****.BXZ.semd}
\end{itemize}

\item Define the adjoint source:
\begin{itemize}
\item Use \texttt{adj\_seismogram.f90}
\item Edit to update \texttt{NSTEP}, \texttt{nrec}, \texttt{t0}, \texttt{deltat}, and the position of the cut to pick
any given phase if needed (\texttt{tstart},\texttt{tend}), add the right number of stations, and
put one component of the source to zero if needed.
\item The output files of \texttt{adj\_seismogram.f90} are \texttt{AA.S****.BXX.adj} and \texttt{AA.S****.BXZ.adj}, for P-SV waves (and
\texttt{AA.S****.BXY.adj}, for SH (membrane) waves). Note that you will need these three
files (\texttt{AA.S****.BXX.adj}, \texttt{AA.S****.BXY.adj} and \texttt{AA.S****.BXZ.adj}) to be present in the \texttt{SEM/} directory
together with the \texttt{absorb\_elastic\_****.bin} and \texttt{lastframe\_elastic.bin} files to be read
when running the adjoint simulation.
\end{itemize}

\item Run the adjoint simulation:
\begin{itemize}
\item Make sure that the adjoint source files absorbing boundaries and last frame files are
in the \texttt{OUTPUT\_FILES/} directory.
\item \texttt{SIMULATION\_TYPE = 3}
\item \texttt{SAVE\_FORWARD = .false.}
\end{itemize}

Output files (for example for the elastic case):
\begin{itemize}
\item \texttt{snapshot\_rho\_kappa\_mu*****}
\item \texttt{snapshot\_rhop\_alpha\_beta*****}
\end{itemize}
which are the primary moduli kernels and the phase velocities kernels respectively, in ascii format
and at the local level, that is as ``\texttt{kernels(i,j,ispec)}''.

\end{enumerate}

\section{Remarks about adjoint runs and solving inverse problems}

% This from Carl Tape:
SPECFEM2D can produce the gradient of the misfit function for a
tomographic inversion, but options for using the gradient within an
iterative inversion are left to the user (e.g., conjugate-gradient,
steepest descent). The plan is to include some examples in the future.

% This from Yang Luo:
The algorithm is simple:
%
\begin{enumerate}
\item calculate the forward wave field $\mathbf{s}(x,t)$
\item calculate the adjoint wave field $\mathbf{s}^\dagger(x,t)$
\item calculate their interaction $\mathbf{s}(x,t) \cdot \mathbf{s}^\dagger(x,T-t)$ (these symbolic, temporal and spatial derivatives should be included)
\item integrate the interactions, which is summation in the code.
\end{enumerate}
%
That is all. Step 3 has some tricks in implementation, but which can be skipped by regular users.

If you look into SPECFEM2D, besides ``\texttt{rhop\_ac\_kl}'' and ``\texttt{rho\_ac\_kl}'',
there are more variables such as ``\texttt{kappa\_ac\_kl}'' and ``\texttt{rho\_el\_kl}'' etc.
``\texttt{rho}'' denotes density $\rho$ (``\texttt{kappa}'' for bulk modulus $\kappa$ etc.),
``\texttt{ac}'' denotes acoustic (``\texttt{el}'' for elastic),
``\texttt{kl}'' means kernel (and you may find ``\texttt{k}'' as well, which is the interaction at each time step, i.e., before doing time integration).

\section{Caution}

Please note that:
%
\begin{itemize}
\item at the moment, adjoint simulations do not support anisotropy, attenuation, and viscous damping.

\item you will need \texttt{AA.S****.BXX.adj}, \texttt{AA.S****.BXY.adj} and \texttt{AA.S****.BXZ.adj}
to be present in directory \texttt{SEM/} even if you are just running an acoustic or
poroelastic adjoint simulation.
  \begin{itemize}
    \item \texttt{AA.S****.BXX.adj} is the only relevant component for an acoustic case.
    \item \texttt{AA.S****.BXX.adj} and \texttt{AA.S****.BXZ.adj} are the only relevant components for a
poroelastic case.
  \end{itemize}
\end{itemize}


