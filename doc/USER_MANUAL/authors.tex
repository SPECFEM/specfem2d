\section*{Authors}

The SPECFEM2D package was first developed by Dimitri
Komatitsch and Jean-Pierre Vilotte at Institut de Physique du Globe
(IPGP) in Paris, France from 1995 to 1997 and then by Dimitri Komatitsch
at Harvard University (USA), Caltech (USA) and then CNRS and University of Pau (France) from 1998 to 2005.
The story started on April 4, 1995, when Prof. Yvon Maday from CNRS and University of Paris, France, gave a lecture to
Dimitri Komatitsch and Jean-Pierre Vilotte at IPG about the nice properties of the Legendre spectral-element method with diagonal mass matrix that he had used for
other equations. We are deeply indebted and thankful to him for that.
That followed a visit by Dimitri Komatitsch to OGS (Istituto Nazionale di Oceanografia e di Geofisica Sperimentale) in Trieste, Italy, in February 1995
to meet with G\'eza Seriani and Enrico Priolo, who introduced him to their 2D Chebyshev version of the spectral-element method with a non-diagonal mass matrix.
We are deeply indebted and thankful to them for that.\newline

%%%%%%%% please do NOT merge the lines of the paragraph below, it makes it difficult to insert new names in alphabetical order; leave a single name per line
%%%%%%%% please do NOT merge the lines of the paragraph below, it makes it difficult to insert new names in alphabetical order; leave a single name per line
%%%%%%%% please do NOT merge the lines of the paragraph below, it makes it difficult to insert new names in alphabetical order; leave a single name per line
Since then it has been developed and maintained by a development team: in alphabetical order,
\'Etienne Bachmann,
Alexis Bottero,
Quentin Brissaud,
Bryant Chow,
Paul Cristini,
Rene Gassmoeller,
Michael Gineste,
Felix Halpaap,
Dimitri Komatitsch,
Jes\'us Labarta,
Matthieu Lefebvre,
Nicolas Le Goff,
Pieyre Le Loher,
Qiancheng Liu,
Qinya Liu,
Youshan Liu,
Zhaolun Liu,
David Luet,
Roland Martin,
Ren\'e Matzen,
Ryan Modrak,
Christina Morency,
Masaru Nagaso,
Daniel Peter,
Eric Rosenkrantz,
Herurisa Rusmanugroho,
Elliott Sales de Andrade,
Carl Tape,
Jeroen Tromp,
Eduardo Valero Cano,
Jean-Pierre Vilotte,
Zhinan Xie,
Zhendong Zhang.\newline

The code is released open-source under the GNU version 3 license, see the license at the end of this manual.

